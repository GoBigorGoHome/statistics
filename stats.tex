\documentclass[12pt,uft8]{ctexrep}
% \usepackage[underscores=false,hybrid]{markdown}
\usepackage{amsmath,amsfonts,amsthm}
\usepackage{xcolor}
\usepackage{physics}
\usepackage[many,documentation]{tcolorbox}
\usepackage{marginnote}
\newcommand{\dif}{\mathop{}\!\mathrm{d}}
\DeclareMathOperator{\Var}{Var}
\DeclareMathOperator{\Cov}{Cov}
\newcommand{\intii}{\int_{-\infty}^\infty}

\begin{document}
格式:

/ 连接的是同一个概念的两个名称

,表示列举

?表示TODO

()表示注解

\chapter{概率论基础}
\section{概念}
\begin{itemize}
\item 概率空间

A probability space is a triple $(\Omega, \mathcal{F}, P)$ where $\Omega$ is a set of "outcomes," $\mathcal{F}$ is a set of "events," and $P\colon \mathcal{F} \to [0,1] $ is a function that assigns probabilities to events. We assume that $\mathcal{F}$ is a $\sigma$-field (or $\sigma$-algebra), i.e., a (nonempty) collection of subsets of $\Omega$ that satisfy

(i) if $A\in\mathcal{F}$ then $A^c \in\mathcal{F}$, and 

(ii) if $A_i\in\mathcal{F}$ is a countable sequence of sets then $\bigcup_iA_i\in\mathcal{F}$.

Here and in what follows, countable means finite or countably infinite. Since $\bigcap_iA_i = \qty(\bigcup_iA_i^c)^c$, it follows that a $\sigma$-field is closed under countable intersections.

Without $P$, $(\Omega,\mathcal{F})$ is called a measurable space, i.e., it is a space on which we can put a measure. A measure is a nonnegative countably additive set of function; that is, a function $\mu\colon\mathcal{F}\to\mathbb{R}$ with\\
(i) $\mu(A)\ge\mu(\emptyset) = 0$ for all $A\in\mathcal{F}$, and\\
(ii) if $A_i\in\mathcal{F}$ is a countable sequence of disjoint sets, then
\[
    \mu\qty(\bigcup_iA_i) = \sum_i\mu(A_i)
\]
If $\mu(\Omega) = 1$, we call $\mu$ a probability measure. In this book, probability measures are usually denoted by $P$.
\item 随机试验/随机现象
\item 样本空间,样本点
\item 随机事件 abbr. 事件
\item 事件 $A$ 与 $B$ 相互独立
\item $n$ 个事件相互独立,两两独立
\item 随机变量 abbr. RV: $X$,$Y$,……

A real valued function $X$ defined on $\Omega$ is said to be a random variable if for every Borel set $B\in\mathbb R$ we have $X^{-1}(B) = \qty{\omega\colon X(\omega)\in B}\in\mathcal{F}$. When we need to emphasize the $\sigma$-field, we will say that $X$ is $\mathcal{F}$-measurable or write $X\in\mathcal{F}$.(A Borel set is an element of a Borel sigma-algebra.)

这个定义我还不能完全理解,我不理解 Borel set 究竟是什么。我本科概统教材上给出的随机变量的定义是:

设 $\Omega$ 为一个样本空间,若对任意 $\omega\in\Omega$,都有一个实数 $X(\omega)$ 与之对应,则称 $X(\omega)$ 为一个随机变量,并简记为 $X$ 。

\item 分布函数/DF $F(x) := P(X\le x)\quad x\in\mathbb R$
\item 概率密度函数 abbr. 密度函数/PDF(在下文中,对于连续 RV,“分布”一词一般指概率密度函数)
\end{itemize}

\section{随机变量的数字特征}

\begin{itemize}
\item 数学期望 abbr. 期望:$E(X)$

\emph{和的期望等于期望的和,不论独立不独立。}

\item $k$ 阶原点矩: $E(X^k)$
\item $k$ 阶中心矩:$E\qty{[X-E(X)]^k}$
\item 方差:$\Var(X) := E\qty{[X-E(X)]^2} =\boxed{\color{blue}{ E(X^2) - [E(X)]^2 }}$(二阶中心矩)
\item 标准差:$\sigma_X = \sqrt{\Var(X)}$
\item $X$,$Y$ 的协方差:

$\Cov(X,Y) := E\qty{[X-E(X)] [Y-E(Y)]} = \boxed{\color{blue}E(XY) - E(X)E(Y)}$
\begin{align*}
\Var(X+Y)& = E[(X+Y)^2] - [E(X+Y)]^2 \\
&= E(X^2) - [E(X)]^2 + E(Y^2) - [E(Y)]^2 + 2[E(XY)- E(X)E(Y)] \\
&= \Var(X) + \Var(Y) + 2\Cov(X,Y) 
\end{align*}
一般地,有
\[\Var{aX+bY} = a^2\Var(X) + b^2\Var(Y) + 2ab\Cov(X,Y)\]
\item $X$,$Y$ 的线性相关系数:$\rho_{XY} = \dfrac{\Cov(X,Y)}{\sqrt{\Var(X)}\sqrt{\Var(Y)}}$
\item $X$ 的标准化随机变量:$X^* := \dfrac{X-E(X)} {\sqrt{\Var(X)}}$
\item $n$ 个随机变量的协方差矩阵:$\Sigma := (\sigma_{ij})_{n\times n}$,$\sigma_{ij} = \Cov(X_i,X_j)$

\item 二维随机变量 $(X,Y)$(可以理解为两个随机变量)
\item $(X,Y)$ 的联合分布函数 $F(X,Y) := P(X\le x, Y\le y)$ 
\item 边际分布函数:$F_X(x)$,$F_Y(y)$
\item 随机变量的函数的分布:$X$ 的 PDF 为 $f(x)$,$Y=g(X)$,求 $Y$ 的 PDF。

\end{itemize}

\section{重要分布}
\subsection{正态分布}
$X\sim N(\mu,\sigma^2)\quad f(x) = \dfrac{1}{\sqrt{2\pi}\sigma} e^{-\dfrac{(x-\mu)^2}{2\sigma^2}}, x\in\mathbb R, \mu\in\mathbb R, \sigma > 0 $
\subsubsection{高斯积分}
考虑广义积分 
$$ A = \int_{-\infty}^{\infty} \dfrac{1}{\sqrt{2\pi}\sigma} e^{-\dfrac{(x-\mu)^2}{2\sigma^2}} \dif x$$
做变量替换,令 $ t = \frac{x - \mu} {\sqrt2\sigma}$,得
$$ A = \frac1{\sqrt\pi} \boxed{\color{blue} {\int_{-\infty}^{\infty} e^{-t^2} \dif t} } $$
$\int_{-\infty}^{\infty} e^{-t^2} \dif t$ 称作高斯积分,也称概率积分。需要证明

$$ I = \int_{-\infty}^{\infty} e^{-t^2} \dif t = \sqrt{\pi} $$
为证此式,考虑
$$ I^2 = \int_{-\infty}^{\infty} e^{-t^2} \dif t  \int_{-\infty}^{\infty} e^{-u^2} \dif u = \int_{-\infty}^{\infty} \int_{-\infty}^{\infty} e^{-\qty(t^2+u^2)}\dif t\dif u $$ 
转化成极坐标 $ t = r\cos\theta, u = r\sin\theta$,上式转化为
$$ I^2 = \int_0^{2\pi}\dif\theta\int_0^\infty e^{-r^2}r\dif r = \pi,$$
明所欲证。
\subsubsection{正态分布的期望与方差}
设 RV $X\sim N(\mu,\sigma^2)$。$X$ 的期望为 
\[
    E(X) = \dfrac{1}{\sqrt{2\pi}\sigma} \int_{-\infty}^\infty x e^{-\dfrac{(x-\mu)^2}{2\sigma^2}} \dif x
\]
做变量替换,令 $u = \frac{x-\mu}{\sqrt2\sigma}$,则 $x = \sqrt2\sigma u+\mu, \dd{x} = \sqrt2\sigma\dd{u}$。代入上式,得
\[
    E(X) = \frac1{\sqrt{\pi}}\intii\qty(\sqrt2\sigma u+\mu)e^{-u^2}\dd{u} = \mu
\]
$X$ 的方差为
\[
    \Var(X) =  \dfrac{1}{\sqrt{2\pi}\sigma} \intii \qty(x-\mu)^2 e^{-\dfrac{(x-\mu)^2}{2\sigma^2}} \dif x
\]
令 $u = \frac{x-\mu}{\sqrt2\sigma}$ 转化成
\[\Var(X) =  \dfrac{2\sigma^2}{\sqrt\pi} \boxed{\color{blue}\intii u^2 e^{-u^2} \dif u}\]
把 $u^2e^{-u^2}\dif u$ 写为 $-\frac12u\dif(e^{-u^2})$,用分部积分,得到
\[\intii u^2 e^{-u^2} \dif u =-\frac12 \intii \dif\qty(ue^{-u^2})-e^{-u^2}\dif u   = \frac12 \intii e^{-u^2}\dif u = \frac{\sqrt\pi}{2} \tcbdocmarginnote{$ue^{-u^2}$是奇函数} \]
于是得到
\[\Var(X) = \sigma^2\]
\begin{tcolorbox}[title=总结]
对于形如 
\[I = \int_0^\infty e^{-u^m}u^n\dif u\qq{$m,n$是正整数}\]
的积分,可用上述分部积分的技巧计算。
\end{tcolorbox}
% 考虑积分
% \[I = \intii u^2 e^{-u^2} \dif u\]
\subsection{泊松分布}
$P(X = k) = \dfrac{\lambda^k}{k!}e^{-\lambda}$,记做 $X\sim P(\lambda)$,$\lambda > 0$ 。
\begin{align*}
    E(X) &= e^{-\lambda}\sum_{k\ge 0} k\dfrac{\lambda^k}{k!} = e^{-\lambda}\sum_{k\ge 1} k\dfrac{\lambda^k}{k!} = \lambda e^{-\lambda}\sum_{k\ge 1} \dfrac{\lambda^{k-1}}{(k-1)!} =  \lambda e^{-\lambda}\sum_{k\ge 0} \dfrac{\lambda^{k}}{k!} = \lambda \\
    E(X^2) &= e^{-\lambda} \sum_{k>=0} k^2   \dfrac{\lambda^k}{k!} = \lambda e^{-\lambda} \sum_{k\ge 0} (k+1) \dfrac{\lambda^{k}}{k!} = \lambda(\lambda + 1)
\end{align*}
从而 $ \Var(X) = E(X^2) - [E(X)]^2 = \lambda $ 。


\chapter{数理统计基础}
\section{概念}
\begin{itemize}
\item 总体,个体,个体的数量指标(\emph{个体的出现是随机的} $\implies$ 个体的数量指标是随机变量,记做 $X$)
\item 抽样
\item 样本,样本容量 $n$(样本是一个复数(plural)概念,抽到的个体是随机得到的,其数量指标是 $n$ 个随机变量 $X_1, \dots, X_n$)
\item 样本容量 $n$,样本观测值 $x_1, \dots x_n$ 。
\item 简单随机抽样,简单随机样本
\item 统计量:函数 $T = T(X_1, \dots, X_n)$,其中不含未知参数。
\item 样本均值:$\overline{X} = \frac1n \sum_{1\le i\le n} X_i$
\item 样本方差:$S^2 = \frac{1}{n-1}\sum_{1\le i\le n} \qty(X_i - \overline{X})^2$,样本标准差 $S$
\item 样本 $k$ 阶原点矩:$A_k = \frac{1}{n}\sum_{1\le i\le n} X_i^k$
\item 极大次序统计量:$X_{(n)} = \max\qty{X_1, \dots, X_n}$
\item 极小次序统计量:$X_{(1)} = \min\qty{X_1, \dots, X_n}$
\item 抽样分布:统计量的分布(统计量也是一个随机变量)
\end{itemize}

\section{三大分布}
\subsection{$\chi^2$分布}
 $\chi^2 = X_1^2 + \dots + X_n^2, \quad X_i\sim N(0,1)$

\subsubsection{推导}
设 RV $X\sim \chi^2(n)$ 。考虑 $X$ 的 DF
$$ P(X \le x)  = \frac1{\qty(\sqrt{2\pi})^n} \int_{V} e^{-\frac12(\sum_i x_i^2)}  \dif x_1 \dif x_2 \dots \dif x_n $$
其中积分区域 $V$ 为 $n$ 维球 $\sum_i x_i^2 \le x$ 。

由于积分区域和被积函数具有球对称性,上述积分在 $n$ 维球坐标系下表示为 
$$ P(X \le x)  = c_n \int_{0}^{\sqrt{x}} e^{-\frac{r^2}{2}}  r^{n-1}\dif r $$
其中 $c_n$ 是与 $n$ 有关的常数。考虑上述积分在 $x\to \infty$ 时的极限,有
\begin{equation}
    1 = c_n \boxed{\color{blue}{\int_{0}^{\infty} e^{-\frac{r^2}{2}}  r^{n-1}\dif r}} \label{Int:chi2}
\end{equation}
形如 $ \int_{0}^{\infty} e^{-\frac{r^2}{2}}  r^{n-1}\dif r $ 的广义积分没有解析形式,引入一种特殊函数来表示这一类积分。

\subsubsection{$\Gamma$ 函数}
$$ \Gamma(x) = \int_0^\infty t^{x-1} e^{-t} \dif t , \quad x > 0 $$
注意:并非对任意 $x\in\mathbb{R}$ 上述广义积分都收敛,$\Gamma(0)$ 就不收敛。

对 \eqref{Int:chi2} 右侧的积分做变量替换,令 $u = r^2/2$ ,则 $r = (2u)^{\frac12}, \dif r = (2u)^{-\frac12}\dif u$ 。于是
$$ \int_{0}^{\infty} e^{-\frac{r^2}{2}}  r^{n-1}\dif r   = \int_{0}^{\infty} e^{-u}  (2u)^{n/2-1} \dif u = 2^{n/2-1} \Gamma\qty(\frac n2)$$
于是 
\[
c_n = \frac1{2^{n/2-1} \Gamma\qty(\frac n2)}
\]
从而 $\chi^2(n)$ 的 PDF 为
$$
    f(x) = \frac{\dif P(X\le x)}{\dif x} = c_n \frac {  e^{-\frac{x}{2}}  x^{\frac{n-1}{2}} \dif\sqrt{x} } {\dif x} = \frac1{2^{n/2} \Gamma\qty(\frac n2)}  e^{-\frac{x}{2}}  x^{\frac{n}{2}-1}
$$

\subsection{$t$分布}
    $X\sim N(0,1)$,$Y\sim \chi^2(n)$ 且 $X,Y$ 独立, $Z$ 定义为
    \[Z = \frac{X}{\sqrt{\frac{Y}{n}}}\]
    $Z$ 的分布称为\emph{自由度为 $n$ 的 $t$ 分布}。

\subsubsection{p.d.f.}
\[
    \Pr(Z\le z)=\frac{1}{2^{n/2}\Gamma(n/2)}\frac{1}{\sqrt{2\pi}} \int_0^\infty \dd{y}e^{-\frac{y}{2}}  y^{\frac{n}{2}-1}\int_{-\infty}^{z\sqrt{y/n}}\dd{x} e^{-x^2/2}
\]
对 $z$ 求导数,得 
\begin{align}
    \dv{\Pr(Z\le z)}{z} &= \frac{1}{2^{n/2}\Gamma(n/2)}\frac{1}{\sqrt{2\pi}} \int_0^\infty \dd{y}e^{-\frac{y}{2}}  y^{\frac{n}{2}-1} \qty(e^{-\frac{z^2y}{2n}} \sqrt{y/n}) \notag\\
&= \frac{1}{2^{n/2}\Gamma(n/2)}\frac{1}{\sqrt{2\pi}}\frac{1}{\sqrt{n}} \int_0^\infty \dd{y}e^{-\frac{z^2+n}{2n}y}  y^{\frac{n-1}{2}}\label{Int:t1} \tcbdocmarginnote{往$\Gamma$函数上凑}
\end{align}
作变量代换,令 $u = \frac{z^2+n}{2n}y$,则 $ y = \frac{2n}{z^2+n} u $,上面的积分变为
\[\qty(\frac{2n}{z^2+n})^{\frac{n+1}{2}}\Gamma(\frac{n+1}{2})\]
以此代入 \eqref{Int:t1},并略加整理,即得 
\begin{align*} \dv{\Pr(Z\le z)}{z} &= \frac{\Gamma\qty((n+1)/2)}{\Gamma(n/2)}\frac{1}{\sqrt{n\pi}}\qty(\frac{n}{z^2+n})^{(n+1)/2} \\
    &= \frac{\Gamma\qty((n+1)/2)}{\Gamma(n/2)}\frac{1}{\sqrt{n\pi}}\qty(1+\frac{z^2}{n})^{-(n+1)/2}
\end{align*}



\begin{tcolorbox}[title=积分号下求导数]
设
\[F(y) = \int_a^b f(x, y)\dd{x}\]
则
\[\dv{F(y)}{y} = \int_a^b \pdv{f(x, y)}{y}\dd{x} \]    
\end{tcolorbox}

\subsection{F分布}
$X\sim \chi^2(n)$,$Y\sim\chi^2(m)$,且 $X,Y$ 独立。称
\[
    Z = \frac{X/n}{Y/m}
\]
服从自由度为 $n,m$ 的 $F$ 分布。

\subsubsection{p.d.f.}
\[\Pr(Z\le z) = \int_0^\infty k_m(y)\dd{y} \int_0^{\frac{ny}{m}z} k_n(x)\dd{x} \]
其中 $k_n(x)$ 表示 $\chi^2(n)$ 的 p.d.f.,于是 $Z$ 的 p.d.f. 为
\begin{align}
    f_{nm}(z) &= \int_0^\infty k_m(y)\dd{y}\qty(\frac{ny}{m} k_n\qty(\frac{nz}{m}y)) \notag\\
    &= \frac nm \qty(\frac{nz}{m})^{n/2-1} K_nK_m \int_0^\infty e^{-\qty(1+\frac{nz}{m})y/2} y^{(m+n)/2-1} \dd{y} \label{Int:F1}
\end{align}
其中,$K_n = \qty(2^{n/2}\Gamma(n/2))^{-1}$ 表示 $\chi^2(n)$ 分布的p.d.f.中的系数。做变量替换,令 $u = \qty(1+\frac{nz}{m})y/2$ 则 $ y = \frac{2m}{m+nz} u$,稍加整理,上面的积分变为
\[ 
    \qty(\frac{2m}{m+nz} )^{(m+n)/2} \int_0^\infty e^{-u} u^{(m+n)/2-1}\dd{u} =   \qty(\frac{2m}{m+nz} )^{(m+n)/2}\Gamma\qty((m+n)/2)
\]
将各项代回 \eqref{Int:F1},得
\[
    f_{nm}(z) = \frac{\Gamma\qty((m+n)/2)}{\Gamma(m/2)\Gamma(n/2)}
n^{n/2} m^{m/2} z^{n/2-1} (m+nz)^{-(n+m)/2}
\]
\section{次序统计量}


% \begin{markdown}

% 设 RV $X$ 的 PDF 为 $f(x)$,$Y$ 的 PDF 为 $g(y)$,求 $Z=X+Y$ 的 PDF $h(z)$,考虑下面几种解法是否正确。

% (1) $ h(z) = \int_{-\infty}^\infty f(x)g(z-x)\dif x$

% (2) $ h(z) = \int_{-\infty}^\infty f(x)g_{Y\mid X}(z-x\mid x)\dif x$

% 其中 $g_{Y\mid X}(z-x\mid x)$ 为给定 $X=x$ 的条件下 $Y$ 的条件密度函数,$g_{Y\mid X}(y\mid x) = \dfrac{f(x,y)}{f_X(x)}$ 。
% 注意:这里的 $f(x,y)$ 不能由 $f(x)$ 和 $g(y)$ 算出来,需要另外给出。
% \end{markdown}
\end{document}